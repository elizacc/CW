\documentclass[a4paper, 12pt]{article}

\usepackage{cmap} % поиск в PDF
\usepackage[english, russian]{babel} % локализация и переносы
\parskip=3pt % дополнительное расстояние между абзацами
\usepackage{graphicx} % вставка рисунков
\usepackage{lastpage}
\usepackage{rotating}
%\usepackage{minted} % красивый код
\usepackage{verbatim}
\usepackage{multirow} % Слияние строк в таблице
\usepackage{caption}
\usepackage{amsfonts, amssymb, amsthm, mathtools, amsmath} % AMS
\usepackage{array} % Дополнительная работа с таблицами
\usepackage{multicol}

%%% Цветной текст

\usepackage[usenames]{color}
\usepackage{colortbl}

%% Поля

\usepackage{geometry} 
\geometry{left=2cm}
\geometry{right=2cm}
\geometry{top=2cm}
\geometry{bottom=2cm}

%%% Гиперссылки

\usepackage[unicode]{hyperref}
\usepackage{xcolor}
\definecolor{urlcolor}{HTML}{4682B4} % цвет гиперссылок
\definecolor{linkcolor}{HTML}{4682B4} % цвет ссылок
\definecolor{citecolor}{HTML}{4682B4} % цвет библиоссылок
\hypersetup{pdfstartview=FitH,  linkcolor=linkcolor,urlcolor=urlcolor, citecolor=citecolor, colorlinks=true}

%% Немного дизайна

\definecolor{lb}{rgb}{0.8,0.85,1}
\renewcommand{\labelitemi}{$\diamond$}

\newcommand{\e}{\mathbb{E}}
\newcommand{\p}{\mathbb{P}}
\newcommand{\n}{\mathbb{N}}
\newcommand{\id}{\mathbb{I}}
\newcommand{\re}{\mathbb{R}}
\DeclareMathOperator{\plim}{plim}
\DeclareMathOperator{\var}{Var}
\DeclareMathOperator{\svar}{sVar}
\DeclareMathOperator{\argmin}{argmin}
\renewcommand{\epsilon}{\varepsilon}
\newcommand{\msum}{\sum\limits_1^n}
\newcommand{\isum}{\sum\limits_i^n}
\newcommand{\jsum}{\sum\limits_j^n}


\begin{document}

\thispagestyle{empty}
\begin{center}
	\textbf{ПРАВИТЕЛЬСТВО РОССИЙСКОЙ ФЕДЕРАЦИИ}\\
	\vspace{3ex}
	\textbf{Федеральное государственное автономное\\ образовательное учреждение высшего образования}
	
	\vspace{3ex}
	
	\textbf{Национальный исследовательский университет \\ <<Высшая школа экономики>>}
	
	\vspace{10ex}
	\begin{flushright}
		Факультет экономических наук\\
		Образовательная программа <<Экономика>>
	\end{flushright}
\end{center}
\vspace{12ex}

\begin{center}
	{\textbf{КУРСОВАЯ РАБОТА
	}}
	\vspace{1ex}
	
	<<Методы интерпретации моделей машинного обучения>>
\end{center}
\vspace{4ex}
\begin{flushright}
	\noindent
	Студентка группы БЭК171\\Махнева Елизавета Александровна\\
	\vspace{13ex}
	Научный руководитель:\\
	Соколов Евгений Андреевич
	
\end{flushright}	

\vfill

\begin{center}
	Москва 2020
	
\end{center}
\newpage
	\tableofcontents
	\newpage
	
	\section{Введение}
	Машинное обучение может найти свое применение практически во всех сферах деятельности человека. Модели помогают обнаруживать мошеннические операции, предсказывать диагноз пациента и многое другое. Однако с развитием машинного обучения создаваемые модели становятся все более сложными. Они показывают высокое качество, однако перестают быть понятными для человека.

Данный недостаток оказывается важным с разных точек зрения. В первую очередь, мы теряем возможность объяснить, как именно модель приняла то или иное решение. Мы не понимаем модель, а значит, не можем контролировать ее обучение напрямую -- лишь перестраивая ее структуру, надеясь изменить результат в нужную сторону.

% здесь недостатки отсутствия интерпретируемости

%Во введении я хочу сказать про то что существует интерпретируемые и нет модели. Про то что нам хотелось бы интерпретировать и привести аргументы за (как минимум те 3 из презенташки). Наверное стоит объединить с блоком интерпретации, так как здесь особо больше нечего писать (если только много не получится), а воду лить не хочется

%1. Интерпретируемые и нет модели. Желательно показать примеры, почему не интерпретируются.
%2. Интерпретация нужна! Потому что...
%3. Плюсы интерпретации и небольшие минусы
%4. Что и как хотелось бы интерпретировать. Основы интерпретации -- какой она должна быть
%5. Кратко перечислить методы, сказать какие они бывают
	\newpage
	
	\section{Методы интерпретации}
	\subsection{PDP}
	\textbf{PDP (Partial Dependence Plot, график частичной зависимости)} -- график, который показывает зависимость прогноза модели от значения отдельного признака. С его помощью мы можем понять, как некоторый признак влияет на результат работы модели. В двумерном случае: по оси ординат изображается прогноз, по оси абсцисс -- анализируемый признак.

Например, <пример>

Идея: визуализация -- это отличный способ интерпретации. Если мы хотим понять, как признаки влияют на результат, то идеальный вариант -- посмотреть, как меняется прогноз от изменения всех признаков одновременно. Однако мы сталкиваемся с проблемой: если признаков больше двух, график построить не получится. Поэтому чтобы сохранить возможность визуализации, мы будем анализировать зависимость результата от одного-двух признаков -- получим PDP.

	\subsubsection{Принцип работы}
	Обозначения:\\
$X = (x_1, \ldots, x_d)$ -- матрица признаков\\
$x_1, x_2$ -- векторы исследуемых признаков\\
$X_b = (x_3, \ldots, x_d)$ -- векторы остальных признаков\\
$a(x_1, \ldots, x_d)$ -- предсказания модели как функция от признаков

Нам нужно получить функцию зависимости предсказания от одного-двух признаков при зафиксированных остальных: $g(x_1, x_2) = a(x_1, x_2 | \, x_3, \ldots, x_d)$. Но если $x_1$ и/или $x_2$ зависимы с признаками из $X_b$, то возникает проблема. При изменении анализируемого признака меняется и зависимый с ним, который мы не рассматриваем -- мы не сможем рассмотреть чистый предельный эффект одного признака, на него всегда будет наложен эффект другого предиктора. Поэтому одной из предпосылок метода является независимость исследуемых признаков от остальных.

Но даже с предпосылкой о независимости признаков функция $g(x_1, x_2)$ не будет показывать точный результат, так как предельные эффекты предикторов разные для разных объектов выборки. Поскольку нашей задачей является посмотреть влияние выбранных признаков в целом, мы рассмотрим, как влияют анализируемые признаки на среднее предсказание. То есть найдем матожидание предсказания модели при фиксированных исследуемых признаках (как констант с точки зрения матожидания):

\[
\bar{g}(x_1, x_2) = \e(a(x_1, x_2, X_b) \, | X_b)
\]

Таким образом, мы получим функцию, которая показывает предельные эффекты признаков для среднего предсказания. Но чтобы найти матожидание, мы должны знать истинные распределения признаков. Поскольку нам недоступна данная информация, можно воспользоваться методом Монте-Карло, чтобы примерно оценить искомую функцию:

\[
\hat{g}(x_1, x_2) = \frac{1}{n} \isum a(x_1, x_2, X_b^{(i)}),
\]
где $X_b^{(i)}$ -- $i$ строка матрицы $X_b$
% не очень корректно, так как в функции векторы и скаляры

Результат: функция показывает, как исследуемые признаки в среднем влияют на результат работы модели. Мы можем построить ее график, чтобы более наглядно посмотреть на влияние предикторов на предсказание.

% добавить преимущества и недостатки

%Признаки могут иметь как непрерывное, так и дискретное распределение -- в том и другом случае мы сможем найти матожидание, при условии, что оно существует. Особо стоит отметить категориальные признаки



% почему бы просто эту функцию не находить
% способ дает меньшую точность там, где зависимость можно было достать из модели
% можно делать анимацию, если нельзя рассмотреть все сразу :)
% Должна быть предпосылка не о независимости случайных величин, а о том, что производная по исследуемому признаку не зависит от других величин -- возможно просто стоит написать, что обычно в моделях комбинации признаков не используются, а если и используются, то задаются извне пользователем. И если в этом способе используются какие-то комбинации даже этого признака с самим собой, и при этом они рассматриваются отдельно, то мне кажется этот метод будет бесполезен
% можно дополнить, что возможно стоит брать не матожидание, а предоставлять подставлять некоторые значения, потому что предельные эффекты все равно будут разные для всех
	\subsubsection{Реализация}
	Реализации данных графиков есть в разных библиотеках: $\verb|pdpbox|$, $\verb|sklearn|$,\\ $\verb|(sklearn.inspection.plot_partial_dependence|$,\\ $\verb|sklearn.ensemble.partial_dependence.plot_partial_dependence)|$

Библиотека $\verb|pdpbox|$ предоставляет более аккуратное и более наглядное представление графиков. Инструкции по установке можно найти \href{https://github.com/SauceCat/PDPbox}{здесь}.

Попробуем самостоятельно построить несколько графиков. Для этого мы будем использовать \href{https://www.kaggle.com/ruchi798/movies-on-netflix-prime-video-hulu-and-disney}{датасет}, содержащий информацию о различных фильмах и сериалах, выходивших с 1900 года и по настоящее время на разных платформах. До построения графиков сделаем предобработку.

%\begin{minted}{Python}
%import pdpbox
%\end{minted}
	\subsection{LIME}
	LIME (Local Interpretable Model-Agnostic Explanations) -- метод, показывающий вклад признаков в отдельное предсказание, работающий с любой моделью.
% нужен ли перевод
% добавить пример работы

\subsubsection{Идея}
Результаты некоторых моделей легко интерпретировать. Например, в линейной регрессии можно посмотреть на веса. Они показывают, насколько изменится предсказание при изменении признаков. Так для каждого конкретного предсказания можно понять, почему модель выдала именно такой результат -- виден непосредственный вклад каждого признака.

Но не все модели легко интерпретировать. Например, некоторые архитектуры нейронных сетей. Они зачастую значительно превосходят линейные модели, но при этом сама структура модели представляет собой <<черный ящик>> -- непонятно, как именно модель сформировала предсказание, какие признаки сильнее повлияли на решение нейронной сети.

Идея состоит в том, чтобы перенести свойство интерпретируемости простых моделей на более сложные. Мы можем обучить интерпретируемую модель по выборке, где ответами являются предсказания сложной модели. В процессе обучения модель анализирует зависимости непосредственно между признаками и предсказаниями сложной модели. Тогда мы сможем интерпретировать результаты простой модели, которые являются аппроксимацией предсказаний сложной модели.

Возникает проблема: сложная модель выявляет зависимости, которые, например, линейная модель может не уловить. Но мы можем воспользоваться свойством, что дифференцируемые функции можно линеаризовать в окрестности заданной точки. То есть, если мы будем рассматривать одно предсказание, то в его небольшой окрестности мы можем считать простую модель аппроксимацией более сложной.

% почему для любой модели - это можно будет в каких-нибудь выводах написать
% почему не вытаскивать из самих моделей?
% добавить картинку про локалити

<Свой пример для чиселок>
	\subsubsection{Принцип работы}
	% авторы в статье вводят очень много обозначений. Нужно попробовать снизить их количество. Можно убрать: G
У нас есть модель $f: \re^d \rightarrow \re$, предсказания которой мы хотим интерпретировать. Пусть $x \in \re^d$ -- векторное представление предсказания, которое мы хотим интерпретировать, $x' \in \{0,1\}^{d'}$ -- интерпретация предсказания в виде бинарного вектора. 

Мы хотим найти модель $g$ из класса интерпретируемых моделей $G$. Область определения $g$: $\{0,1\}^{d'}$. Стоит отметить, что сложность моделей обычно обратно зависит от ее интерпретируемости. Например, линейную модель с 2-3 признаками гораздо проще интерпретировать, чем модель с 10 и более признаками. Поэтому чтобы не терять интерпретируемость модели при ее приближении к более сложной, нужно ввести меру сложности $\Omega(g)$ как регуляризацию в нашей задаче. %Сложность модели может ограничиваться пользователем при регулировании гиперпараметров: глубина решающего дерева, количество моделей в ансамблях, количество слоев в нейронной сети и т.д.
% может быть здесь стоит писать все про интерпретируемые модели и не приводить такие примеры. Вообще не уверена что это нужна писать, здесь же мат часть
% все-таки не до конца поняла почему такая область определения у g, надо разобраться

Введем меру близости $\pi_x(z)$ между объектами $x$ и $z$, чтобы определить окрестность рядом с $x$, внутри которой мы можем использовать простую модель. И наконец определим нашу функцию потерь, которую мы будем оптимизировать: $L(f, g, \pi_x)$ -- разница между моделями $f$ и $g$ в окрестности, заданной $\pi_x$. Тогда в целом задача алгоритма выглядит следующим образом:
\[
explanation(x) = \xi(x) = \underset{g \in G}{\argmin} (L(f, g, \pi_x) + \Omega(g))
\]

Одной из особенностей алгоритма является его независимость от модели, которую необходимо интерпретировать. Поэтому мы не можем приписывать модели $f$ никакие свойства. Вместо этого мы будем аппроксимировать ее, искусственно создавая объекты в окрестности $x$ и получая для них предсказания из $f$.


%Понятие окрестности носит абстрактный характер, поэтому чтобы учесть расстояние между объектами на практике, мы будем использовать меру близости $\pi_x(z)$ как вес, с которым $z$ влияет на функцию потерь.
	\subsubsection{Реализация}
	Данный метод реализован в библиотеке $\verb|lime|$. Она содержит в себе методы для работы с разными типами данных: $\verb|lime.lime_tabular|$, $\verb|lime.lime_text|$, $\verb|lime.lime-image|$. Инструкцию по установке можно найти \href{https://github.com/marcotcr/lime}{здесь}.
	\subsection{SHAP}
	SHAP (SHapley Additive exPlanations) -- метод, оценивающий вклад признаков в предсказания модели на основе значений Шэпли.

\subsubsection{Идея}
Предсказание модели формируется на основе признаков объектов. Если мы хотим узнать влияние отдельного признака, мы можем построить предсказание модели без него и с ним и посмотреть как меняется результат. % почему реально не убирать насовсем?
Но модель может быть слишком сложной, чтобы мы могли оценить влияние предиктора по одному объекту, регулируя один признак при фиксированных остальных. % а корректно ли занулять его, это ведь тоже какое-то значение
% помнить про то, что у shapley values есть свой раздел
Правильнее рассмотреть все возможные комбинации всех признаков: их разные значения, наличие/отсутствие, чтобы понять, как в каждом из перечисленных случаев добавление и исключение признака влияет на предсказание. Но рассматривая влияние в каждом конкретном случае, мы получаем огромное количество предельных эффектов, что тяжело интерпретируется. Поэтому можно рассмотреть, какой в среднем оказывает эффект включение признака в модель.

Мысли, которые нужно включить сюда:

1. Как считается value

2. Есть value, есть вклад, есть прогноз -- надо понять как вклад связан с value

3. Есть интерпретация -- можно вывести из предпосылок формулу value

4. Я не хочу выводить этот алгоритм через аддитивные модели -- я хочу попроще написать, а потом указать что она аддитивная и показать, что это

5. По возможности расписать как их находить и что с ними потом делать для интерпретации

6. Расписать разницу SHAP и Shaply values -- пока что не очень понятно
% я хочу привести так, чтобы вообще не пришлось ссылаться при объяснении на теорию игр, просто потом сказать что так совпадает
% привести к тому, что они показывают отклонение от среднего -- объясняют его
%Здесь на помощь приходит понятие из теории игр -- значения Шэпли (Shapley values). Они показывают, какой вклад внес в общий выигрыш отдельный игрок при кооперации. В нашей задаче мы рассматриваем предсказание модели как выигрыш, а признаки как игроков. 
	\subsubsection{Shapley values (значения Шэпли)}
	\input{chapters/3_shapval.tex}
	\subsubsection{Принцип работы}
	Метод SHAP несколько отличается от использования значений Шэпли напрямую для интерпретации вклада признаков -- он модифицирует классический подход. Рассмотрим SHAP подробнее.

Вместо использования большого количества моделей мы обучим одну модель и далее будем пользоваться только ее предсказаниями. Тогда наша модель представима в виде $f: \re^d \rightarrow \re$. Вместо исключения из нее признаков (в таком случае нам пришлось бы переобучать модель) мы оставим данные признаки в виде случайных величин и посчитаем математическое ожидание предсказания при фиксированных включенных в модель признаках.

Для упрощения расчетов мы не будем рассматривать разные значения предикторов, а бинаризуем их представление, обозначив за <<1>> наличие предиктора и за <<0>> его отсутствие: $x \rightarrow x', \, x' = \{1\}^d$. % не знаю корректно ли
Обозначим за $z' \in \{0,1\}^d$ объект, у которого мы учитываем только признаки из $S$ при формировании предсказания:
$z_j =
\begin{cases}
1, \text{если $j \in S$}\\
0, \text{если $j \notin S$}\\
\end{cases}$.\\
То есть $z'$ -- одна из возможных комбинаций предикторов.

Обученная модель работает с исходным видом признаков, поэтому нам нужно также восстанавливать значения исходных признаков по бинарному вектору. Введем для этого функцию $h_x(z') = z$, где $x$ -- исходный вектор исследуемого объекта, $z'$ -- бинарный вектор, в котором некоторые признаки заменены нулями, $z$ -- представление вектора $z'$ в исходном пространстве признаков. Функция $h_x$ вместо единиц восстанавливает значения из вектора $x$, а вместо нулей оставляет признак как некоторую переменную (случайную величину).
% наверняка можно как-то умнее упрощать признаки
Тогда наше ожидаемое предсказание для $z$ представимо в виде математического ожидания:
\[
\bar{f}(z) = \e(f(z)|\,z_S)
\]

Мы знаем конкретные значения признаков $z_S$, так как функция $h_x$ перенесла их из объекта $x$. Найдя математическое ожидание мы можем подставить данные значения, чтобы получить условное математическое ожидание, которое и будет оценкой нашего предсказания. И уже данную формулу мы можем использовать при расчете значений Шэпли.% здесь для его расчета мы почему-то переходили к безусловному матожиданию, это связано со свойствами значений Шэпли

Но данные расчеты все еще довольно-таки трудоемкие, так как нам нужно перебрать все возможные комбинации признаков. Даже при обучении одной модели это затратно. Поэтому чтобы еще сильнее упростить расчеты мы воспользуемся свойствами, которыми обладают значения Шэпли. % если понадобится, то лучше все-таки вывести данные свойства отдельно. в целом алгоритм понятен, они тоже задают некоторые веса, тоже обучают модель, и тоже получают веса как значения

Мысли, которые нужно включить сюда:

+1. Как считается value

+2. Есть value, есть вклад, есть прогноз -- надо понять как вклад связан с value

+3. Есть интерпретация -- можно вывести из предпосылок формулу value

4. Я не хочу выводить этот алгоритм через аддитивные модели -- я хочу попроще написать, а потом указать что она аддитивная и показать, что это

5. По возможности расписать как их находить и что с ними потом делать для интерпретации

6. Расписать разницу SHAP и Shaply values -- пока что не очень понятно

% привести к тому, что они показывают отклонение от среднего -- объясняют его -- это осталось из основного источника
% еще там прописан монте-карло опять
	\subsubsection{Реализация}
	\input{chapters/3_2shap.tex}
	\newpage

	\section{Данные и модели}
	Рассмотрим описанные методы на конкретной задаче. Построим относительно сложную модель (градиентный бустинг) для бинарной классификации объектов.
	\subsection{Данные}
	Для анализа был выбран датасет, содержащий информацию о клиентах разных отелей за 2015-2017 года. Задача: предсказать, отменит ли клиент бронь. Датасет был предварительно обработан:
\begin{itemize}
	\item Удалены переменные company (много пропусков), reservation status, reservation status date (данная информация обычно бывает известна уже после отмены либо отъезда гостя)
	\item Приведены к числовому виду порядковые переменные meal, deposit type, arrival date month
	\item Удалены строки с пропущенными значениями -- таких строк было немного и в них была пропущена важная информация
\end{itemize}

Отдельно стоит отметить, что категориальные переменные reserved room type, assigned room type, hotel, country, market segment, distribution channel, customer type не были приведены к числовому виду, так как далее (спойлер) будет использоваться модель CatBoost, которая самостоятельно обрабатывает такие признаки.
	\subsection{Модели}
	Для анализа описанных методов я выбрала модель бинарной классификации XGBoost. Подбор гиперпараметров осуществлялся с помощью кросс-валидации на тренировочных данных. Одной из проблем данных является их несбалансированность: она решалась учетом объектов положительного класса с большим весом. Посмотреть на процесс кросс-валидации и обучения модели можно \href{https://github.com/elizacc/CW}{здесь}
% исправить ссылку
	\subsection{Попытка интерпретации}
	Модель достигла неплохого качества в задаче классификации: ROC-AUC=0.9, accuracy=0.82. Попробуем интерпретировать ее результаты.

\textbf{Нулевой способ это встроенный метод XGBoost}, показывающий важность признаков при предсказании. Посмотрим на первые 5 самых важных признаков. Данная величина может быть расчитана тремя способами: <<weight>>, <<gain>>, <<cover>>. Первый показывает, сколько раз признак появляется в дереве:

\begin{figure}[h]
	\centering{\includegraphics[width=0.85\linewidth]{pics/imp1.png}}
\end{figure}

Второй -- на сколько в среднем уменьшалась ошибка при использовании данного признака:

\begin{figure}[h]
	\centering{\includegraphics[width=0.85\linewidth]{pics/imp2.png}}
\end{figure}

И последний -- какое количество объектов выборки задействовало узлы с заданным признаком:

\begin{figure}[h]
	\centering{\includegraphics[width=0.85\linewidth]{pics/imp3.png}}
\end{figure}

Теперь перейдем к описанным ранее методам. \textbf{Первый -- PDP}. Возьмем признаки, которые сам XGBoost посчитал наиболее важными: первые два из weight (adr, lead\_time), первый из gain (deposit\_type) и первый из cover (distribution\_channel\_GDS) -- они с отрывом вырываются в лидеры.

И также возьмем признаки, которые XGBoost счел самыми незначительными: последний из weight (distribution\_channel\_GDS, забавно -- в тренировочной выборке всего 145 объектов, которым соответствует GDS. Судя по всему данный признак встречается 1-2 раза в узлах деревьев, но при этом он отсекает очень много объектов, из-за чего cover считает его важным), последний из gain (customer\_type\_group) и последний из cover (market\_segment\_Offline TA/TO).

Построим для них PDP.

\begin{tabular}{c|c}
	\arrayrulecolor[rgb]{0.8,0.85,1}
	\includegraphics*[width = 0.47\textwidth]{pics/mypdp1.png} & \includegraphics*[width = 0.47\textwidth]{pics/mypdp2.png}\\
	\hline
	\includegraphics*[width = 0.47\textwidth]{pics/mypdp3.png} & \includegraphics*[width = 0.47\textwidth]{pics/mypdp4.png}\\
	\hline
	\includegraphics*[width = 0.47\textwidth]{pics/mypdp5.png} & \includegraphics*[width = 0.47\textwidth]{pics/mypdp6.png}\\
\end{tabular}\\[2mm]

Первый признак является непрерывным, второй --  дискретным, третий -- порядковым, остальные -- бинарными. Для первых двух видно распределение значений признака (штриховка под графиком) и коридор, показывающий как данный признак влиял на разные объекты в выборке. Для остальных: выборка была кластеризована, и для каждого кластера из 1000 была построена усредненная по подвыборке кривая.

По графикам видно, что:
\begin{itemize}
	\item признак adr оказался важным с точки зрения PDP -- для больших значений (> 50) он вносит положительный вклад в прогноз, увеличивая вероятность отмены брони. Но видно, что для некоторых объектов в выборке он оказывал также и отрицательное влияние -- коридор задевает область отрицательных значений.
	
	Данный признак показывает, сколько в среднем гость тратит на проживание и связанные с ним расходы. Исходя из графика, можно сделать вывод, что чем больше предстоящие расходы, тем выше вероятность отмены брони -- звучит логично, клиент вероятнее отменит бронь, если для него эта поездка окажется слишком дорогой. Причем вклад данного признака стабилизируется с ростом затрат и составляет +0.2 к вероятности отмены в среднем
	
	\item Аналогичный график у lead time (время от открытия брони до приезда).
	
	Здесь ситуация также интуитивно понятна: если гость очень заранее забронировал номер, то за время до приезда его планы могут поменяться. Поэтому клиент вероятнее отменит бронь в данном случае -- +0.2-0.4 к вероятности отмены
	
	\item deposit type также оказался важным признаком: в среднем он оказывает положительное воздействие на предсказание, которое доходит вплоть до полного влияния в виде +0.9 к вероятности. Ни для одного кластера признак не оказывает отрицательное воздействие, однако возможно для отдельных объектов это неверно -- важно аккуратно интерпретировать результаты.
	
	Здесь результат несколько контринтуитивен. Если у клиента есть полный предоплаченный депозит, который не возвращается, то вероятность отмены брони стремится к единице, что нелогично -- внесенный залог должен мотивировать гостей приезжать. Далее мы видим, что для большинства кластеров при переходе к возвращаемому частичному депозиту вероятность практически не меняется и остается около единицы -- это уже более логично, однако также спорно: в случае отмены придется тратить время на бюрократию, связанную с возвратом средств. Но для части кластеров при переходе к возвращаемому депозиту вероятность отмены даже падает, что вызывает сомнения в корректности использования данного признака. Возможно, стоило выбрать другую форму для данного признака: сделать его бинарным, а не порядковым -- то есть, возможно, данная ситуация сложилась из-за неправильного представления категорий депозита.
	
	\item distribution channel GDS отрицательно влияет на предсказание, причем довольно-таки сильно: при переходе от 0 к 1 вероятность отмены брони снижается в среднем на 0.1, максимально по кластерам на 0.3
	
	Данный признак показывает то, что гость воспользовался глобальной системой бронирования, то есть вероятно самостоятельно организовал себе поездку. Снижение вероятности в данном случае не очень логично: человеку проще отменить поездку, когда он ее организовал сам. Также в таком случае выше шанс возникновения проблем в поездке (по сравнению с ораганизацией, предоставляемой туристическими агентствами), из-за чего бронь также может быть отменена
	
	\item у customer type group похожий график, однако влияние существенно ниже. Вероятность снижается на 0.04 в среднем, максимально по кластерам всего лишь на 0.1
	
	График показывает, что при организации групповой поездки ниже шанс, что она отменится. Исходно не очень понятно: если бронь на группу, то она вероятнее отменится, потому что сложно организовать путешествие на целую группу, или она вероятнее не отменится из-за, например, обязательства каждого перед другими (чтобы не ставить людей из своей группы в неудобное положение). То есть данный график приносит некоторое дополнение к нашим данным. Теперь мы знаем, что если поездка организуется для группы, то бронь скорее не будет отменена -- учитываем данное дополнение с осторожностью, учитывая неточность метода
	
	\item признак market segment offline TA/TO оказался незначительным, в среднем он вносит нулевой вклад в предсказание. Однако по кластерам видно, что разброс существенный. То есть данный признак оказывает влияние на каждый отдельный объект -- PDP в виде усредненной кривой не совсем корректный выбор для интерпретации в данном случае, так как он не учитывает разброс
\end{itemize}

Построим также графики взаимодействия признаков (PDP для двух признаков)\footnote{При построении второго типа графика (contour), который показывает линии уровня может возникнуть ошибка. Это связано с несоответствием версии библиотеки matplotlib}. Посмотрим на совместное влияние признаков, которые XGBoost по критерию weight счел важными: adr и lead time.

\begin{tabular}{c|c}
	\arrayrulecolor[rgb]{0.8,0.85,1}
	\includegraphics*[width = 0.5\textwidth]{pics/mypdp7.png} & \includegraphics*[width = 0.42\textwidth]{pics/mypdp8.png}\\
\end{tabular}\\[2mm]

Данный график показывает не влияние предикторов, а непосредственно предсказание. Действительно, взаимодействие данных признаков меняет картину (то есть предсказание не линейно зависит от них). Разброс значений наблюдается с 0.199 до 0.772, разница в 0.573. Особо выделяется пик, который сложился именно из взаимодействия -- желтый островок со значениями 0.7+

В целом мы видим, что при росте времени до приезда и затрат вероятность растет. И выделяется особый случай, когда затраты не очень большие, а время до приезда велико -- в таком случае вероятность отмены брони максимальна.
\newpage
Также интересно узнать, станет ли market segment offline TA/TO более значимым в сочетании с другим, например, с lead time.

\begin{tabular}{c|c}
	\arrayrulecolor[rgb]{0.8,0.85,1}
	\includegraphics*[width = 0.44\textwidth]{pics/mypdp9.png} & \includegraphics*[width = 0.44\textwidth]{pics/mypdp10.png}\\
\end{tabular}\\[2mm]

Рассмотрение двух признаков привело к появлению влияния market segment offline TA/TO -- при больших значениях lead time он становится более значимым, его предельный эффект при lead time $=709$ в среднем составляет $0.572-0.633=-0.06$, что по модулю больше нуля. При малых значениях lead time он перестает влиять на предсказание. Это отличается от результатов, полученных ранее.

Получили добавку к интерпретации market segment offline TA/TO -- если поездка организована туроператором, то она менее вероятно отменится, что интуитивно понятно.

С помощью PDP мы убедились в корректности восприятия моделью некоторых признаков. Для других -- в контринтуитивности результатов, а также нашли возможность для пересмотра формата признаков с целью улучшения модели. Стоит отметить, что в PDP есть очевидный недостаток -- если признак равен 0, то его влияние также равно 0, что не всегда правда. Важно обращать на это внимание при интерпретации.

\textbf{Перейдем к следующему методу -- LIME}. Рассмотрим конкретную поездку (первая из тренировочной выборки), чтобы увидеть, какие признаки оказались наиболее важными для предсказания отмены брони. Истинное значение для данного объекта: 0, то есть бронь не была отменена. 

\begin{figure}[h]
	\centering{\includegraphics[width=0.7\linewidth]{pics/mylime1.png}}
\end{figure}

Модель предсказала, что бронь будет отменена, и LIME с некоторой погрешностью показывает нам, что на это повлияло. Мы видим, что отсутствие отмен ранее снизило вероятность текущей отмены, что корректно. Снова столкнулись с признаком deposit type -- отсутствие депозита также снизило вероятность отмены -- в целом не противоречит ничему, однако отсутствие депозита позволяет без каких-либо потерь отменить бронь, то есть все же скорее увеличивает вероятность отмены с точки зрения интуиции. Два данных признака сильнее всего повлияли на снижение вероятности отмены: -0.52 от каждого.

Клиент не запросил парковочное место, что увеличило вероятность отмены причем довольно-таки сильно (+0.35) -- неочевидное влияние. Отсутствие не отмененных ранее бронирований также повысило вероятность отмены, что звучит логично. Судя по всему это новый для отеля клиент. Он заказывал номер через онлайн-туроператора, и это повысило вероятность отмены на 0.11 -- интуитивно понятно, по интернету проще отменить бронь.

LIME вывел первые пять признаков, которые сильнее всего повлияли на результат. Остальные признаки довели предсказание до того, что получила модель. Она ошиблась с ответом, и мы по полученному объяснению можем понять почему. Вопросы вызывают признаки deposit type и required car parking spaces -- они оказывают неочевидное воздействие на предсказание, возможно, именно из-за них модель ошиблась. Также стоит обратить внимание на силу влияния признаков: не очень понятно, почему предикторы, снижающие вероятность отмены имеют такое большое значение. То же можно сказать и про required car parking spaces.

Таким образом, LIME позволяет выявить недостатки в понимании смысла признаков у модели. Благодаря этому мы можем предотвратить использование некорректной модели и попытаться улучшить ее.

\textbf{Посмотрим на то же самое предсказание с помощью последнего рассматриваемого метода -- SHAP}.
	\newpage
	
	\section{Анализ результатов}
	\input{chapters/6_analysis.tex}
	\newpage
	
	\section{Заключение}
	Таким образом, существующие методы интерпретации моделей машинного обучения показывают неплохие результаты. Безусловно, они имеют свои недостатки, но несмотря на это они справляются со своей задачей и с некоторой погрешностью объясняют работу сложных моделей.
	\newpage
	
	\begin{thebibliography}{99}
		\bibitem{basis}
		\href{https://christophm.github.io/interpretable-ml-book/}{Interpretable Machine Learning | Christoph Molnar | Christoph Molnar | 2020 | all pages}
		\bibitem{LIME}
		\href{https://arxiv.org/pdf/1602.04938.pdf}{“Why Should I Trust You?” Explaining the Predictions of Any Classifier
| ...}
	\end{thebibliography}
\end{document}