Машинное обучение находит свое применение во многих сферах деятельности человека. Модели помогают обнаруживать мошеннические операции, предсказывать диагноз пациента и многое другое. Однако с развитием машинного обучения создаваемые модели становятся все более сложными. Они показывают высокое качество, но перестают быть понятными для человека.

Данный недостаток оказывается важным с разных точек зрения. В первую очередь, мы теряем возможность объяснить, как именно модель приняла то или иное решение. Мы не понимаем модель, а значит, не можем контролировать ее обучение напрямую -- лишь перестраивая структуру, надеясь изменить результат в нужную сторону. Высокая метрика качества, как и несколько метрик одновременно, также не говорят о корректности модели. Во-первых, они могут противоречить друг другу, из-за сложнее сделать корректный вывод. Во-вторых, по-прежнему есть вероятность, что модель выучит неверную закономерность, которую мы не обнаружим, пока в нашу тестовую выборку не попадет подходящий объект.

Помимо этого, отсутствие интерпретируемости может служить препятствием на пути внедрения моделей в различные области. Если мы не знаем, почему алгоритм, управляющий беспилотным автомобилем, принял решение увеличить скорость перед пешеходным переходом, мы вряд ли начнем массово выпускать данную технику. В подобных областях использование моделей накладывает большие риски -- поэтому важно осознавать, как именно алгоритмы принимают решения. Препятствием также является непонимание со стороны людей, незнакомых с машинным обучением. Человек, незнакомый с направлением, не сможет сразу понять, как работает тот или иной алгоритм. Поэтому создание методов интерпретации ускоряет распространение машинного обучения также за счет снижения порога необходимых для работы с ним знаний.

Модели используются во многих областях, где выгода от них перевешивает риски, которые они несут. Тем не менее, даже если нам нестрашны несколько ошибок, совершенные машиной, мы можем упускать важные детали, которые позволяют улучшить алгоритм и избежать увеличения количества ошибок в будущем. Например, в случае смещенности выборки модель может дискриминировать некоторые типы объектов: при скрининге резюме -- отказывать женщинам, при выдаче кредита -- учитывать расу клиента и т.д. Закономерность, заложенная в алгоритм, некорректна. Практически невозможно выявить данную проблему без понимания того, как модель создает свое предсказание.

Интерпретация не только дает возможность проверить корректность модели, но и в паре с машинным обучением может помочь извлечь дополнительную информацию из имеющихся данных. Если мы обучим модель, а затем интерпретируем результаты ее работы, то сможем посмотреть на наши данные с другой точки зрения: какие признаки оказались важны для предсказания, какие закономерности присутствуют в выборке и т.д. Данный бонус дает некоторое понимание устройства мира. Причем его можно использовать и с целью получения новых знаний: извлечение существующих эмпирических закономерностей в природе и обществе.

Безусловно, интерпретация имеет и некоторые недостатки. Два основных -- невозможность достичь абсолютной точности интерпретации и необходимость жертвовать качеством модели. Первый интуитивно понятен: мы никогда не можем быть уверены, что корректно интерпретируем <<черный ящик>> модели. Даже прозрачные алгоритмы интерпретируются с некоторой неточностью: например, в линейной регрессии веса не всегда могут соответствовать вкладу признака -- данное правило нарушается при наличии зависимости между признаками. Второй минус: зачастую методы интерпретации применяются к сложным, непрозрачным моделям. Поскольку сами по себе они неинтерпретируемы, приходится применять методы, которые некоторым образом воздействуют на них и зачастую снижают их качество, упрощают их.

Поэтому важно понимать, какая стоит задача в исследовании: получить высокое качество или корректную модель. В первом случае можно обойтись без интерпретации. Во втором случае интерпретация -- полезный инструмент, проверяющий адекватность модель.

Как раз для задач, в которых важно понимание, существует огромное количество методов интерпретации. В данной работе мы рассмотрим три из них, что, безусловно, не является исчерпывающим списком.