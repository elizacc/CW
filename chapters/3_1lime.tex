% авторы в статье вводят очень много обозначений. Нужно попробовать снизить их количество. Можно убрать: G
У нас есть модель $f: \re^d \rightarrow \re$, предсказания которой мы хотим интерпретировать. Пусть $x \in \re^d$ -- векторное представление предсказания, которое мы хотим интерпретировать, $x' \in \{0,1\}^{d'}$ -- интерпретация предсказания в виде бинарного вектора. 

Мы хотим найти модель $g$ из класса интерпретируемых моделей $G$. Область определения $g$: $\{0,1\}^{d'}$. Стоит отметить, что сложность моделей обычно обратно зависит от ее интерпретируемости. Например, линейную модель с 2-3 признаками гораздо проще интерпретировать, чем модель с 10 и более признаками. Поэтому чтобы не терять интерпретируемость модели при ее приближении к более сложной, нужно ввести меру сложности $\Omega(g)$ как регуляризацию в нашей задаче. %Сложность модели может ограничиваться пользователем при регулировании гиперпараметров: глубина решающего дерева, количество моделей в ансамблях, количество слоев в нейронной сети и т.д.
% может быть здесь стоит писать все про интерпретируемые модели и не приводить такие примеры. Вообще не уверена что это нужна писать, здесь же мат часть
% все-таки не до конца поняла почему такая область определения у g, надо разобраться

Введем меру близости $\pi_x(z)$ между объектами $x$ и $z$, чтобы определить окрестность рядом с $x$, внутри которой мы можем использовать простую модель. И наконец определим нашу функцию потерь, которую мы будем оптимизировать: $L(f, g, \pi_x)$ -- разница между моделями $f$ и $g$ в окрестности, заданной $\pi_x$. Тогда в целом задача алгоритма выглядит следующим образом:
\[
explanation(x) = \xi(x) = \underset{g \in G}{\argmin} (L(f, g, \pi_x) + \Omega(g))
\]

Одной из особенностей алгоритма является его независимость от модели, которую необходимо интерпретировать. Поэтому мы не можем приписывать модели $f$ никакие свойства. Вместо этого мы будем аппроксимировать ее, искусственно создавая объекты в окрестности $x$ и получая для них предсказания из $f$.


%Понятие окрестности носит абстрактный характер, поэтому чтобы учесть расстояние между объектами на практике, мы будем использовать меру близости $\pi_x(z)$ как вес, с которым $z$ влияет на функцию потерь.