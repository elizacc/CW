Мы рассмотрели три метода интерпретации моделей машинного обучения: PDP, LIME, SHAP. Они имеют строгое математическое обоснование: простые, но красивые идеи, которые нашли свое применение в машинном обучении. Важным общим свойством методов является использование графиков и изображений -- наиболее понятная, хорошо интерпретируемая человеком форма информации.

Существует огромное количество других методов, помимо приведенных, которые также хорошо интерпретируют работу модели. В последнее время данная область развивается особенно активно, так как распространение использования машинного обучения вызвало рост спроса на объяснение и интерпретацию результатов разных техник. 

Описанные методы позволяют интерпретировать результат работы любой модели. Это полезное средство для самых разных исследований. Как мы уже заметили, они имеют свои недостатки, не всегда точно показывают влияние признаков на предсказание. Тем не менее они выполняют основную задачу -- представляют результат работы модели в понятном человеку виде.