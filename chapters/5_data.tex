Для анализа был выбран датасет, содержащий информацию о клиентах разных отелей за 2015-2017 года. Описание датасета можно найти в \hyperref[sec:app1]{приложении}. Задача: предсказать, отменит ли клиент бронь. Датасет был предварительно обработан:
\begin{itemize}
	\item Удалены переменные company (много пропусков), agent (ID туристических агенств -- зависит от приведенной классификации), reservation status, reservation status date (данная информация обычно бывает известна уже после отмены либо отъезда гостя), arrival date year, arrival date month (данные представлены только по 2015-2017 годам, слишком мало информации для предсказаний на более дальние периоды; месяц аналогично несет мало информации -- вместо данных переменных осталась переменная arrival date week number)
	\item Стандартизированы все числовые переменные, кроме arrival date week number, arrival date day of month (порядковые переменные), is repeated guest (бинарная переменная)
	\item Приведены к числовому виду порядковые переменные meal, deposit type, reserved room type, assigned room type
	\item Приведены к бинарному виду категориальные переменные hotel, country, market segment, distribution channel, customer type
	\item Удалены строки с пропущенными значениями -- таких строк было немного и в них была пропущена важная информация
	\item Удален выброс: для одного наблюдения adr $<0$, что не может быть правдой, так как adr $> 0$
\end{itemize}