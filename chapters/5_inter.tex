Модель достигла неплохого качества в задаче классификации: ROC-AUC=0.9, accuracy=0.82. Попробуем интерпретировать ее результаты.

\textbf{Нулевой способ это встроенный метод XGBoost}, показывающий важность признаков при предсказании. Посмотрим на первые 5 самых важных признаков. Данная величина может быть расчитана тремя способами: <<weight>>, <<gain>>, <<cover>>. Первый показывает, сколько раз признак появляется в дереве:

\begin{figure}[h]
	\centering{\includegraphics[width=0.85\linewidth]{pics/imp1.png}}
\end{figure}

Второй -- на сколько в среднем уменьшалась ошибка при использовании данного признака:

\begin{figure}[h]
	\centering{\includegraphics[width=0.85\linewidth]{pics/imp2.png}}
\end{figure}

И последний -- какое количество объектов выборки задействовало узлы с заданным признаком:

\begin{figure}[h]
	\centering{\includegraphics[width=0.85\linewidth]{pics/imp3.png}}
\end{figure}

Теперь перейдем к описанным ранее методам. \textbf{Первый -- PDP}. Возьмем признаки, которые сам XGBoost посчитал наиболее важными: первые два из weight (adr, lead\_time), первый из gain (deposit\_type) и первый из cover (distribution\_channel\_GDS) -- они с отрывом вырываются в лидеры.

И также возьмем признаки, которые XGBoost счел самыми незначительными: последний из weight (distribution\_channel\_GDS, забавно -- в тренировочной выборке всего 145 объектов, которым соответствует GDS. Судя по всему данный признак встречается 1-2 раза в узлах деревьев, но при этом он отсекает очень много объектов, из-за чего cover считает его важным), последний из gain (customer\_type\_group) и последний из cover (market\_segment\_Offline TA/TO).

Построим для них PDP.

\begin{tabular}{c|c}
	\arrayrulecolor[rgb]{0.8,0.85,1}
	\includegraphics*[width = 0.47\textwidth]{pics/mypdp1.png} & \includegraphics*[width = 0.47\textwidth]{pics/mypdp2.png}\\
	\hline
	\includegraphics*[width = 0.47\textwidth]{pics/mypdp3.png} & \includegraphics*[width = 0.47\textwidth]{pics/mypdp4.png}\\
	\hline
	\includegraphics*[width = 0.47\textwidth]{pics/mypdp5.png} & \includegraphics*[width = 0.47\textwidth]{pics/mypdp6.png}\\
\end{tabular}\\[2mm]

Первый признак является непрерывным, второй --  дискретным, третий -- порядковым, остальные -- бинарными. Для первых двух видно распределение значений признака (штриховка под графиком) и коридор, показывающий как данный признак влиял на разные объекты в выборке. Для остальных: выборка была кластеризована, и для каждого кластера из 1000 была построена усредненная по подвыборке кривая.

По графикам видно, что:
\begin{itemize}
	\item признак adr оказался важным с точки зрения PDP -- для больших значений (> 50) он вносит положительный вклад в прогноз, увеличивая вероятность отмены брони. Но видно, что для некоторых объектов в выборке он оказывал также и отрицательное влияние -- коридор задевает область отрицательных значений.
	
	Данный признак показывает, сколько в среднем гость тратит на проживание и связанные с ним расходы. Исходя из графика, можно сделать вывод, что чем больше предстоящие расходы, тем выше вероятность отмены брони -- звучит логично, клиент вероятнее отменит бронь, если для него эта поездка окажется слишком дорогой. Причем вклад данного признака стабилизируется с ростом затрат и составляет +0.2 к вероятности отмены в среднем
	
	\item Аналогичный график у lead time (время от открытия брони до приезда).
	
	Здесь ситуация также интуитивно понятна: если гость очень заранее забронировал номер, то за время до приезда его планы могут поменяться. Поэтому клиент вероятнее отменит бронь в данном случае -- +0.2-0.4 к вероятности отмены
	
	\item deposit type также оказался важным признаком: в среднем он оказывает положительное воздействие на предсказание, которое доходит вплоть до полного влияния в виде +0.9 к вероятности. Ни для одного кластера признак не оказывает отрицательное воздействие, однако возможно для отдельных объектов это неверно -- важно аккуратно интерпретировать результаты.
	
	Здесь результат несколько контринтуитивен. Если у клиента есть полный предоплаченный депозит, который не возвращается, то вероятность отмены брони стремится к единице, что нелогично -- внесенный залог должен мотивировать гостей приезжать. Далее мы видим, что для большинства кластеров при переходе к возвращаемому частичному депозиту вероятность практически не меняется и остается около единицы -- это уже более логично, однако также спорно: в случае отмены придется тратить время на бюрократию, связанную с возвратом средств. Но для части кластеров при переходе к возвращаемому депозиту вероятность отмены даже падает, что вызывает сомнения в корректности использования данного признака. Возможно, стоило выбрать другую форму для данного признака: сделать его бинарным, а не порядковым -- то есть, возможно, данная ситуация сложилась из-за неправильного представления категорий депозита.
	
	\item distribution channel GDS отрицательно влияет на предсказание, причем довольно-таки сильно: при переходе от 0 к 1 вероятность отмены брони снижается в среднем на 0.1, максимально по кластерам на 0.3
	
	Данный признак показывает то, что гость воспользовался глобальной системой бронирования, то есть вероятно самостоятельно организовал себе поездку. Снижение вероятности в данном случае не очень логично: человеку проще отменить поездку, когда он ее организовал сам. Также в таком случае выше шанс возникновения проблем в поездке (по сравнению с ораганизацией, предоставляемой туристическими агентствами), из-за чего бронь также может быть отменена
	
	\item у customer type group похожий график, однако влияние существенно ниже. Вероятность снижается на 0.04 в среднем, максимально по кластерам всего лишь на 0.1
	
	График показывает, что при организации групповой поездки ниже шанс, что она отменится. Исходно не очень понятно: если бронь на группу, то она вероятнее отменится, потому что сложно организовать путешествие на целую группу, или она вероятнее не отменится из-за, например, обязательства каждого перед другими (чтобы не ставить людей из своей группы в неудобное положение). То есть данный график приносит некоторое дополнение к нашим данным. Теперь мы знаем, что если поездка организуется для группы, то бронь скорее не будет отменена -- учитываем данное дополнение с осторожностью, учитывая неточность метода
	
	\item признак market segment offline TA/TO оказался незначительным, в среднем он вносит нулевой вклад в предсказание. Однако по кластерам видно, что разброс существенный. То есть данный признак оказывает влияние на каждый отдельный объект -- PDP в виде усредненной кривой не совсем корректный выбор для интерпретации в данном случае, так как он не учитывает разброс
\end{itemize}

Построим также графики взаимодействия признаков (PDP для двух признаков)\footnote{При построении второго типа графика (contour), который показывает линии уровня может возникнуть ошибка. Это связано с несоответствием версии библиотеки matplotlib}. Посмотрим на совместное влияние признаков, которые XGBoost по критерию weight счел важными: adr и lead time.

\begin{tabular}{c|c}
	\arrayrulecolor[rgb]{0.8,0.85,1}
	\includegraphics*[width = 0.5\textwidth]{pics/mypdp7.png} & \includegraphics*[width = 0.42\textwidth]{pics/mypdp8.png}\\
\end{tabular}\\[2mm]

Данный график показывает не влияние предикторов, а непосредственно предсказание. Действительно, взаимодействие данных признаков меняет картину (то есть предсказание не линейно зависит от них). Разброс значений наблюдается с 0.199 до 0.772, разница в 0.573. Особо выделяется пик, который сложился именно из взаимодействия -- желтый островок со значениями 0.7+

В целом мы видим, что при росте времени до приезда и затрат вероятность растет. И выделяется особый случай, когда затраты не очень большие, а время до приезда велико -- в таком случае вероятность отмены брони максимальна.
\newpage
Также интересно узнать, станет ли market segment offline TA/TO более значимым в сочетании с другим, например, с lead time.

\begin{tabular}{c|c}
	\arrayrulecolor[rgb]{0.8,0.85,1}
	\includegraphics*[width = 0.44\textwidth]{pics/mypdp9.png} & \includegraphics*[width = 0.44\textwidth]{pics/mypdp10.png}\\
\end{tabular}\\[1mm]

Рассмотрение двух признаков привело к появлению влияния market segment offline TA/TO -- при больших значениях lead time он становится более значимым, его предельный эффект при lead time $=709$ в среднем составляет $0.572-0.633=-0.06$, что по модулю больше нуля. При малых значениях lead time он перестает влиять на предсказание. Это отличается от результатов, полученных ранее.

Получили добавку к интерпретации market segment offline TA/TO -- если поездка организована туроператором, то она менее вероятно отменится, что интуитивно понятно.

С помощью PDP мы убедились в корректности восприятия моделью некоторых признаков. Для других -- в контринтуитивности результатов, а также нашли возможность для пересмотра формата признаков с целью улучшения модели. Стоит отметить, что в PDP есть очевидный недостаток -- если признак равен 0, то его влияние также равно 0, что не всегда правда. Важно обращать на это внимание при интерпретации.

\textbf{Перейдем к следующему методу -- LIME}. Рассмотрим конкретную поездку (первая из тренировочной выборки), чтобы увидеть, какие признаки оказались наиболее важными для предсказания отмены брони. Истинное значение для данного объекта: 0, то есть бронь не была отменена. 

\vspace{-2mm}
\begin{figure}[h]
	\centering{\includegraphics[width=0.7\linewidth]{pics/mylime1.png}}
\end{figure}
\vspace{-2mm}

Модель предсказала, что бронь будет отменена, и LIME с некоторой погрешностью показывает нам, что на это повлияло. Мы видим, что отсутствие отмен ранее снизило вероятность текущей отмены, что корректно. Снова столкнулись с признаком deposit type -- отсутствие депозита также снизило вероятность отмены -- в целом не противоречит ничему, однако отсутствие депозита позволяет без каких-либо потерь отменить бронь, то есть все же скорее увеличивает вероятность отмены с точки зрения интуиции. Два данных признака сильнее всего повлияли на снижение вероятности отмены: -0.52 от каждого.

Клиент не запросил парковочное место, что увеличило вероятность отмены причем довольно-таки сильно (+0.35) -- неочевидное влияние. Отсутствие не отмененных ранее бронирований также повысило вероятность отмены, что звучит логично. Судя по всему это новый для отеля клиент. Он заказывал номер через онлайн-туроператора, и это повысило вероятность отмены на 0.11 -- интуитивно понятно, по интернету проще отменить бронь.

LIME вывел первые пять признаков, которые сильнее всего повлияли на результат. Остальные признаки довели предсказание до того, что получила модель. Она ошиблась с ответом, и мы по полученному объяснению можем понять почему. Вопросы вызывают признаки deposit type и required car parking spaces -- они оказывают неочевидное воздействие на предсказание, возможно, именно из-за них модель ошиблась. Также стоит обратить внимание на силу влияния признаков: не очень понятно, почему предикторы, снижающие вероятность отмены имеют такое большое значение. То же можно сказать и про required car parking spaces.

Таким образом, LIME позволяет выявить недостатки в понимании смысла признаков у модели. Благодаря этому мы можем предотвратить использование некорректной модели или попытаться улучшить ее.

\textbf{Посмотрим на то же самое предсказание с помощью последнего рассматриваемого метода -- SHAP}. Ранее мы рассматривали принцип работы SHAP и выявили его сходство с LIME. Но по умолчанию они используют разные линейные модели и веса для расчетов значений признаков. Поэтому признаки, которые они обозначат за наиболее важные, могут отличаться.

\vspace{-1mm}
\begin{figure}[h]
	\centering{\includegraphics[width=\linewidth]{pics/myshap1.png}}
\end{figure}
\vspace{-7mm}

Стоит отметить, что реализация SHAP показывает нам, как модель пришла к предсказанию, отталкиваясь от некоторого начального значения (base value). Все признаки, кроме market segment online TA, не совпадают с теми, что выявил LIME.

С точки зрения SHAP тип зарезервированной комнаты и дополнительные требования от клиента повысили вероятность того, что клиент не отменит бронь. Тип комнаты представлен в зашифрованном виде, поэтому мы не можем сказать, корректно ли модель учитывает признак. Однако наличие дополнительных требований интерпретируется логично, так как предоставление необходимых клиенту услуг безусловно вызывает положительный отклик с его стороны.

В то же время такие признаки, как расходы, связанные с проживанием, бронь через онлайн-туроператора, неделя прибытия и предоставляемый тип комнаты (тот, который будет предоставлен клиенту по факту -- может отличаться по техническим причинам или пожеланию клиента). Первые два уже обсуждались ранее и интуитивно понятны. Последний также мог бы быть опущен, поскольку мы не знаем, как расшифровать признак. Но стоит отметить, что забронированный и назначенный тип комнаты совпали, что выглядит как выполнение требований клиента. Соответственно, это должно снижать вероятность отмены брони. Но модель этого не учитывает, так как не знает взаимосвязи между признаками. В частности поэтому предикторы, связанные с типами комнаты, могли некорректно влиять на предсказание.

Мы смотрели на отдельное предсказание. Попробуем также построить другие типы графиков, чтобы увидеть более общую картину.

\vspace{-1mm}
\begin{figure}[h]
	\centering{\includegraphics[width=\linewidth]{pics/myshap2.png}}
	\caption*{Forceplot для подвыборки: показывает, как получено предсказание для разных объектов}
\end{figure}
\vspace{-4mm}

Мы видим, какие признаки повлияли сильнее всего на предсказание для отмеченного нами объекта, а также его предсказание, 0.4901.

Интерактивный интерфейс позволяет посмотреть различные комбинации. На графике выше показана зависимость предсказания модели от объекта, причем объекты были отсортированы по схожести между собой. Посмотрим на похожие графики:

\vspace{-1mm}
\begin{figure}[h]
	\centering{\includegraphics[width=\linewidth]{pics/myshap3.png}}
	\caption*{Forceplot: объект отсортированы по значению предсказания}
\end{figure}
\vspace{-4mm}

Для выбранного объекта решающими оказались два признака, причем на их основании, <<по мнению SHAP>>, модель сделала уверенный выбор, предсказав вероятность отмены 0.9998.

\vspace{-1mm}
\begin{figure}[h]
	\centering{\includegraphics[width=\linewidth]{pics/myshap4.png}}
	\caption*{Forceplot: объект отсортированы по возрастанию признака lead\_time}
\end{figure}
\vspace{-4mm}

Данный график является аналогом PDP, построенным по методу SHAP -- он показывает, как в среднем меняется предсказание в зависимости от значения признака lead time

\begin{figure}[h]
	\centering{\includegraphics[width=\linewidth]{pics/myshap5.png}}
	\caption*{Forceplot: зависимость эффектов признака adr от признака lead\_time}
\end{figure}

Также можно посмотреть, как меняется эффект от выбранного признака при разных значениях другого -- некоторый аналог PDP для двух признаков. Например, при adr = 81 lead time снижает вероятность отмены

\begin{figure}[h]
	\centering{\includegraphics[width=0.6\linewidth]{pics/myshap6.png}}
	\caption*{Dependence plot для признака: показывает зависимость SHAP value признака от его значения}
\end{figure}

По данному графику можно сделать вывод, что чем меньше значение lead time, тем более значим признак для предсказания. Интересно, что, например, при lead time $=100$ SHAP value, то есть вклад признака в предсказание, может быть как положительным, так и отрицательным. Цвет точек на графике показывает значение другого признака, в данном случае market segment online TA -- он выбирается автоматически, как признак с которым выбранный взаимодействует сильнее всего. В зависимости от значения второго признака, зависимость, показанная на графике может меняться -- она разная для голубых и красных точек в данном случае.

\begin{figure}[h]
	\centering{\includegraphics[width=0.7\linewidth]{pics/myshap7.png}}
	\caption*{Summary plot для всех признаков и объектов: показывает выборочное распределение признаков, их SHAP values}
\end{figure}

Некоторые выводы из данного графика:
\begin{itemize}
	\item deposit type при низких значениях (отсутствие депозита) вносит очень маленький положительный вклад в предсказание, при высоких значениях наоборот большой отрицательный вклад
	\item вклад lead time равномерно распределяется: при высоких значениях -- большой отрицательный вклад, при низких значениях -- большой положительный вклад
	\item required car parking spaces не вносит вклад при маленьких значениях (0), и вносит большой положительный вклад в предсказание при больших значениях
\end{itemize}